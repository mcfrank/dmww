\documentclass[man,noapacite]{apa2}
\usepackage{amsmath}
\usepackage{booktabs}
\usepackage{apacite2}
\usepackage{fullpage,rotating}
\usepackage{pslatex}
\usepackage{amssymb}
% \usepackage{synctex}

\title{Learning word meaning by inferring speakers' intended referents: \\ An incremental approach to socially-guided statistical learning}

\threeauthors{Michael C. Frank}{Molly L. Lewis}{Noah D. Goodman}
\threeaffiliations{Department of Psychology, Stanford University}{Department of Psychology, Stanford University}{Department of Psychology, Stanford University}

\shorttitle{Learning words by inferring reference}
\rightheader{Learning words by inferring reference}


\acknowledgements{Many thanks to ...

~

\noindent Please address correspondence to Michael C. Frank, Department of Psychology, Stanford University, 450 Serra Mall (Jordan Hall), Stanford, CA, 94305, tel: (650) 724-4003, email: \texttt{mcfrank@stanford.edu}.}


\abstract{How do children learn word meanings?}

\begin{document}
\maketitle                            


\section{Introduction}


\section{Model}

\subsection{Model Specification}

\begin{equation}
P(L, I| W, O) \propto P(W, O | L, I) P(L, I)).
\end{equation}

\noindent But the objects $O$ are observed in the context. In addition, for simplicity, we assume that there is a uniform prior over possible intentions (though we return to this issue in the Discussion). By the generative model in Figure \ref{fig:genmod}, the remaining expression can be factored as follows:

\begin{equation}
P(L, I| W, O) \propto P(W | I, L) P(I | O) P(L).
\end{equation}

In this model, the lexicon $L$ consists of two separate parts. The referential lexicon $L_R$ is a set of Dirichlet-Multinomial distributions

\begin{equation}
P(L) =
\end{equation}

\begin{equation}
P(W | I, L) =
\end{equation}


\subsection{Inference}

\subsubsection{Batch inference using a gibbs sampler}

\subsubsection{Incremental inference using a particle filter}

\section{Simulations}

\subsection{Cross-situational word learning with adults}

\subsubsection{Yu \& Smith (2007)}

\subsection{Experiments with children}

\subsubsection{Disambiguation}

\subsubsection{Dewar \& Xu (2007)}


\subsection{Corpus simulations}

\subsection{Rollins subset (Frank, Goodman, \& Tenenbaum, 2009)}

\subsection{Fernald \& Morikawa (Johnson, Demuth, \& Frank, 2012)}

\section{Discussion}

\newpage

\bibliographystyle{apacite}
\bibliography{icom}

\end{document}

